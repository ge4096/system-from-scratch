\documentclass{article}

\usepackage[letterpaper, margin=1.5in]{geometry}
\usepackage[defaultsans]{droidsans}
\usepackage[defaultmono]{droidmono}
\usepackage{amsmath}
\usepackage[hidelinks]{hyperref}
\usepackage{tabularx}
\usepackage{tcolorbox}

\renewcommand{\familydefault}{\sfdefault}

\renewcommand{\arraystretch}{1.15}

\newtcolorbox{codebox}[1]{colback=black!10!white, colframe=black!90!white, arc=0em, adjusted title=#1}


\begin{document}
\vspace*{\fill}
\begin{flushright}
{\Huge\bf The System from Scratch\\Assembler Specification\\}
\vspace{2em}
{\Large Version 0.0.0}
\end{flushright}
\vspace*{\fill}
\pagebreak{}

\tableofcontents
\pagebreak{}

\section{Introduction}
The System from Scratch Assembler (or {\tt sfsasm}) is a program.

\section{Syntax}
\subsection{Literals}

\subsection{Labels}

\subsection{Defines}

\subsection{Instructions}

\subsection{Directives}
In addition to the instructions supported by the System from Scratch ISA, {\tt sfsasm} supports several pseudo-instruction directives for manipulating memory contents within the program being assembled. Directives are similar in syntax to instructions, with a directive name followed by some operands. Each directive name begins with a {\tt .} to distinguish them from actual instructions.

The {\tt .align} directive inserts zero bytes starting from the current location in memory so the start of the next entry in memory will have an address which is a multiple of the operand to {\tt .align}. The operand to {\tt .align} can be an octal, hexadecimal, or positive decimal literal or the name of a define.

\subsection{Comments}
{\tt sfsasm} supports single-line comments. In {\tt sfsasm}, a single-line comment starts with {\tt //} and ends with a new line. Comments can occur anywhere in the code and are ignored by the assembler.

\begin{codebox}{Comments}
\begin{verbatim}
// This is a single-line comment.

jmp my_label; // Comments are ignored by the assembler.

//
// Multi-line comments can be effectively created by
// joining together several single-line comments.
//
\end{verbatim}
\end{codebox}

\begin{verbatim}
my_define = 1
my_other_define = my_define
.data my_other_define
my_define = 2
.data my_other_define
\end{verbatim}
\end{document}
